The following describes how Tanger currently handles rollback data on a
thread's stack. The basic problem is that in LLVM IR, loads/stores are
explicit only if they access memory that has an address. For a thread (and
thus the code executed by the thread), this is memory accessed through
pointers (e.g., on the heap) or memory that is forced to be addressable
using pointers (i.e., allocated by LLVM's \texttt{alloca} instruction).
Explicit loads/stores are easily transformed because Tanger just needs to
transform the respective \texttt{load}/\texttt{store} instructions.

However, there can be more loads and stores in the machine code generated from
the LLVM IR because LLVM assumes to have an unlimited number of registers but
real machines don't have this (the stack is used as temporary storage).

The first approach is to try to add explicit rollback code for the data that is
written to the stack. This requires changing LLVM's code that generates binary
code for each supported architecture (the backends). One could probably come up
with a generic patch for that, but it's not likely that this would become a
part of LLVM currently. Because requiring users to use a patched LLVM is not
good (and the C backend would need another approach anyway), we don't use this
first approach.

The second approach is to perform a bulk stack save/restore on transaction
begin. Consider a thread that is about to begin a transaction. Data on it's
caller's stack frames is not visible and accesses to this data will use
explicit loads/stores. Data on stack frames that were used by callees (e.g.,
if the transaction begin is actually a restart), will not be needed and should
not be rolled back anyway. Thus, we only need rollback for the current
stack frame. LLVM offers intrinsics to get the stack frame's address
(\texttt{llvm.frameaddress}) and intrinsics to save/restore the stack
(\texttt{llvm.stacksave} and \texttt{llvm.stackrestore}), which probably
return/set the stack pointer. If we roll back the stack frame (the area
between stack pointer and frame address), the state for the thread should be
completely restored on a restart.\footnore{\texttt{siglongjmp} restores
registers and signal masks. Explicit memory accesses are rolled back by the
STM. Memory allocation is shared with other threads and the application uses
memory allocation like a black box, so we don't need rollback there.}

This approach is fairly easy to implement and should work on a lot of
architectures. However, it has some (potential) problems.
First, application runtime overheads can be significantly larger in case
of transactions that only modify a small subset of large stack frames, or no
data on the stack at all (which might be a common case).

Second, the code on the path between the transaction begin marker and actual
instructions that restore the stack content could use space on the stack as
well. We try to solve this by letting Tanger insert LLVM code directly at the
begin instead of delegating stack restore to STM functions. Frame
address and stack pointer are obtained directly after the call to
\texttt{sigsetjmp}. Stack save/restore is performed by inserting calls to
\texttt{memcpy} (using LLVM intrinsics), which is probably not going to be
inlined, and if it uses stack space, this will probably be on a new frame that
it outside of the memory area that is restored.

We do ask the STM to return a memory area for stack save/restore between
getting the stack frame's addresses and the actual save/restore instructions.
If this STM function and other STM functions after the save/restore are
inlined, there could be common expressions stored on the stack. However, the
stack area is of no use to other parts of the STM, so it's outcome and
whether it was executed for save or restore probably does not affect the rest
of the STM. We could avoid this potential problem by disallowing inlining of
the STM function that returns the memory area for stack contents.

Also see the discussion of code reordering by the compiler around transaction
begin and end. Architectures that have a different stack concept could also
require a different solution.
